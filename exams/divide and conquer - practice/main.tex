\documentclass[a4paper, 12pt, english]{article}
\usepackage[utf8]{inputenc}
\usepackage{fancyhdr}
\usepackage{graphicx}
\usepackage{lastpage}
\usepackage{layout}
\usepackage{enumitem}
\usepackage{etoolbox}
\usepackage{amsmath}
\usepackage{mathptmx}
\usepackage[bottom]{footmisc}
\usepackage[includeheadfoot, left=3cm, right=3cm, top = 1.5 cm]{geometry}
\usepackage{minted}
\usemintedstyle{xcode}


\graphicspath{ {./answers/} }

\pagestyle{fancy}
\fancyhf{} 
\rhead{
    {\Large \textbf{Analysis and Design of Algorithms}}\\
    \textbf{CS2102} \\ 
    \textbf{Analysis of Algorithms Practice} \\ 
    \textbf{2020-II}
}
\lhead{\includegraphics[width=4.6cm, keepaspectratio]{logo/utec}}
\rfoot{\textbf{\thepage}\hspace{1pt} of \textbf{\pageref{LastPage}}}
\cfoot{}

\setlength{\parindent}{0em}
\setlength{\headheight}{80pt}

\newcounter{problem}[section]
\newenvironment{problem}[3][]{\refstepcounter{problem}\par\medskip 

\textbf{Problem~\theproblem  ~~(#2) - 
\ifboolexpr{
  test {\ifdimless{1 pt}{#3 pt}}
}
{#3 points} % true
{#3 point} % false
} \newline\newline } {\medskip} 


\begin{document}

\textbf{Submission deadline}: 15 Oct, 20:05\\ 
\textbf{Number of questions}: 5

\begin{itemize}
    \item Write your C++ code inside the \emph{answers} folder in order to generate a single PDF file. 
        The \emph{problem\{1,2,3,4,5\}.cpp} files should include the functions required by the problem and the main function. 
        No need to create header or other source files per problem. 
    \item Read the questions carefully and write your answers clearly. Answers that are not legible will not have any score. 
    \item Notes are allowed. To compile this file you can use the command latexmk -pdf -shell-escape main.tex
    \item Consider edge cases properly, any file that doesn't compile or doesn't met the requirements will not have any grade (clang++ or g++ are preferred).
    \item \textbf{STD compiler flag}: -std=c++2a (replace this by your actual configuration)
\end{itemize}

\underline{Outcomes}:

\begin{enumerate}[label=\alph*.]
    \item Apply appropriate mathematical and related knowledge to computer science.
    \item Analyze problems and identify the appropriate computational requirements for its solution.
\end{enumerate}
\noindent\rule{\textwidth}{0.01pt}
\vspace{4mm}

\begin{problem}{Outcomes a}{5}
    Write a divide-and-conquer $\Theta(n*log(n))$ program to measure how far a sequence of $n$ numbers is from being sorted in descending order. \newline 

    \textbf{Intuition}: If we measure the number of changes we need to do in a list of numbers to be sorted in ascending order
    we need to count the number of pairs $i,j$ such that $A[i] > A[j]$ and $i<j$.
    That count will give us the measure of how far a sequence of numbers is from being sorted.
    For example, using the previous definition the similarity measure for the array \{2,4,1,3,5\} is 3 since (2,1), (4,1) and (4,3) are out of order.

    \inputminted[fontsize=\small,breaklines]{cpp}{answers/problem1/problem1.cpp}
\end{problem}

\begin{problem}{Outcomes a}{3}
    Write a divide-and-conquer $\Theta(n*log(n))$ program that counts the total number of times that a number $k$ appears in a sequence of numbers.

    \inputminted[fontsize=\small,breaklines]{cpp}{answers/problem2/problem2.cpp}
\end{problem}

\begin{problem}{Outcomes a}{5}
    Consider the Volatile Chemical Corporation investing problem described in Cormen's Chap 04.
    Write a divide-and-conquer $\Theta(n*log(n))$ program that receives an input array of $n$ numbers representing the stock prices of each day and returns the days that maximizes our profit when buying and selling stocks.

    \inputminted[fontsize=\small,breaklines]{cpp}{answers/problem3/problem3.cpp}
\end{problem}

\begin{problem}{Outcomes a}{3}
    Write a lineal (1pt) and a divide-and-conquer (2pt) $\Theta(log(n))$ program that compute $x^{n}$ considering $x$ as the base and $n$ as the exponent.

    \inputminted[fontsize=\small,breaklines]{cpp}{answers/problem4/problem4.cpp}
\end{problem}

\begin{problem}{Outcomes a, b}{4}
    Write a recursive insertion sort algorithm (2pt) that receives an unsorted array of numbers, then express that algorithm using a recurrence relation and approximate T(n) (2pt). 

    \inputminted[fontsize=\small,breaklines]{cpp}{answers/problem5/problem5.cpp}

    \begin{align*}
        T(n) &= 2T(n/2) + n & \text{(replace these lines with your answer)}\\ 
        T(n) &= O(n)
    \end{align*}
\end{problem}

\end{document}
